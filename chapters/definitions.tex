\section*{Appendix A. Glossary}
\addcontentsline{toc}{part}{Appendix A. Glossary}
\rhead{ \fancyplain{}{GLOSSARY} }

\vspace*{-2em}
{
\small
\setdefaultleftmargin{8pt}{4pt}{}{}{}{}

	\begin{enumerate}
		\item {\large circular motion}
		\begin{compactitem}
			
			\item \emph{radian}: angle subtended at centre of circle by arc equal in length to the radius
			
			\item \emph{angular displacement}: angle through which an object moving along a circular path
			
			\item \emph{angular velocity}: rate of change in angular displacement
			
			%angle swept out by an object moving along a circular path per unit time; rate of change of angular displacement
			
			\item \emph{centripetal acceleration}: acceleration of an object moving in a circle that always points towards the centre of the circle
			
			\item \emph{centripetal force}: resultant force acting on an object moving in a circle, this force always points towards the centre of the circle
			
		\end{compactitem}
		
		\item {\large gravitational fields}
		
		\begin{compactitem}
			
			\item \emph{Newton's law of gravitation}: gravitational force between two \underline{point} objects is directly proportional to the product of their masses, and inversely proportional to the square of their separation ($F \propto \frac{m_1m_2}{r^2}$)
			
			\item \emph{gravitational field}: a region of space in which a mass experiences a force
			
			\item \emph{direction of gravitational field (at a point)/direction of gravitational field line}: direction of force acting on a test mass (at that point)
			
			\item \emph{gravitational field strength}: gravitational force per unit mass $(g=\frac{F}{m})$
			
			\item \emph{gravitational potential (at a point)}: work done by to bring \underline{unit} mass from \underline{infinity} (to this point)%, potential at infinity is defined to be zero
			
		\end{compactitem}
		
		\item {\large oscillations}
		
		\begin{compactitem}
			
			\item \emph{oscillation}: an object moves back and forth about its equilibrium position
			
			\item \emph{simple harmonic motion}: acceleration is proportional to displacement from the equilibrium position, and is opposite to the displacement ($ a = -\omega^2 x$)
			
			\item \emph{free oscillation}: an oscillation at its own natural frequency with no external driving force/no energy gain or loss/constant amplitude
			
			\item \emph{forced oscillation}: an oscillation under an external driving force
			
			\item \emph{kinematic properties of simple harmonics}: maximum speed but zero acceleration at equilibrium position, maximum acceleration but zero speed at maximum displacement
			
			\item \emph{damping}: resistive forces causing gradual decrease in amplitude/loss of total energy
			
			\item \emph{effects of damping}: amplitude/total energy of oscillation decreases exponentially, frequency/period is almost unchanged (or slightly reduced)
			
			\item \emph{resonance}: when frequency of driving force is close to natural frequency of an oscillator, amplitude of oscillation becomes maximum
			
		\end{compactitem}
		
		\item {\large thermal physics}
		
		\begin{compactitem}
			
			\item \emph{internal energy}: sum of the \underline{random} distribution of kinetic energy and potential energy of molecules ($U = E_k + E_p$)
			
			\item \emph{thermal energy}: energy transferred from one object to another because of a difference in their temperatures
			
			\item \emph{thermal equilibrium}: when two objects in contact have the same temperature, there is no net flow of thermal energy
			
			\item \emph{specific heat capacity}: thermal energy required per unit mass to cause a unit change in temperature ($c = \frac{Q}{m\Delta T}$)
			
			\item \emph{specific latent heat}: thermal energy required per unit mass to change the state of matter with no change in temperature ($L = \frac{Q}{m}$)
			
			\item \emph{first law of thermodynamics}: change in internal energy of a thermodynamic systems equals the heat supplied and external work done on the system ($\Delta U = Q + W$)
			
		\end{compactitem}
		
		\item {\large ideal gases}
		
		\begin{compactitem}
			
			\item \emph{ideal gas}: a gas that obeys the equation $pV=nRT$ or $pV=NkT$ at any pressure $p$, volume $V$ and \underline{thermodynamic} temperature $T$
			
			\item \emph{assumptions of ideal gas model}: gas molecules are widely separated, so no intermolecular forces; collisions between molecules are perfectly elastic; volume of molecules is negligible compared to volume occupied by the gas; molecules are in continuous random motion; molecules travel in straight lines between collisions, etc.
			
			%\item \emph{Avogadro's constant ($N_A$)}: the number of atoms in 0.012 kg of a carbon-12 sample
			
			\item \emph{cause of gas pressure}: gas molecules move randomly, as they hit the wall of the container and rebound, the change in momentum gives rise to a force, contribution of many molecules produces a pressure on the container
			
			\item \emph{relating kinetic energy with temperature}: mean kinetic energy of particles of an ideal gas is directly proportional to its thermodynamic temperature ($\left\langle E_k \right\rangle =\frac{3}{2}kT$)
			
		\end{compactitem}
		
		\item {\large electric fields}
		
		\begin{compactitem}
			
			\item \emph{Coulomb's law}: electric force between two \underline{point} objects is proportional to the product of their electric charges, and the inverse square of their separation ($F \propto \frac{q_1q_2}{r^2}$)
			
			\item \emph{electric field}: an area of space in which a charged object experiences a force
			
			\item \emph{direction of electric field (at a point)/direction of electric field line}: direction of force acting on a \underline{positive} charge (at that point)
			
			\item \emph{electric field strength}: electric force per unit \underline{positive} charge ($E=\frac{F}{q}$)
			
			\item \emph{electric potential (at a point)}: work done to bring a unit \underline{positive} charge from \underline{infinity} (to that point)
			
			\item \emph{similarities between gravitational and electric fields}: both forces obey inverse square law; field strength inversely proportional to distance squared; potential inversely proportional to distance; potential is zero at infinity, etc.
			
			\item \emph{differences between gravitational and electric fields}: gravity is always attractive, but electric field can be either attractive or repulsive; gravitational potential is always negative, while electric potential can take both signs, etc.
			
			\item \emph{relating field strength and potential}: field strength equals (negative) gradient of potential ($E=-\frac{\mathrm{d}V}{\mathrm{d}r}$ for electric fields, $g=-\frac{\mathrm{d}\phi}{\mathrm{d}r}$ for gravitational fields)
			
			\item \emph{electric field inside charged conductor/metal}: charge distribution is stable, no electric force on charge carriers/free electrons, so field strength inside conductor must be zero; moving charge inside conductor requires no work done, so potential is constant
			
		\end{compactitem}
		
		\item {\large capacitance}
		
		\begin{compactitem}
			
			\item \emph{capacitance of a parallel-plate capacitor}: ratio of charge on \underline{one} plate of a capacitor to its p.d. between the plates ($C = \frac{Q}{V}$)
			
			\item \emph{function of capacitors}: store and release energy, separate charges, smooth voltages, block d.c. signals, etc.
			
			\item \emph{how a capacitor stores energy}: to charge a capacitor, positive and negative charges are pushed onto each plate, work must be done to separate the charges, this converts to electrical energy stored in capacitor
			
			\item \emph{explaining the decrease in p.d. for a discharging capacitor}: as p.d. across capacitor decreases, discharging current it drives through resistor decreases, rate of discharging decreases, so rate of decrease in p.d. decreases
			
			\item \emph{time constant and rate of charing/discharging}: greater time constant $\tau = RC$ means slower rate of charging/discharging: greater resistance $R$ then  charging/discharging current becomes smaller; greater capacitance $C$ then more charges must be transferred
			
		\end{compactitem}
		
		\item {\large magnetic field}
		
		\begin{compactitem}
			
			\item \emph{magnetic field}: a region of space where a magnet/a moving charge/a current-carrying conductor experiences a force
			
			\item \emph{magnetic flux density}: magnetic force per unit length acting on a wire carrying unit current at \underline{right angle} to the magnetic field ($B=\frac{F}{Il}$)
			
			%\item \emph{right-hand rule}: a rule to determine direction of magnetic field produced by a current-carrying conductor
			
			%\item \emph{Fleming's left-hand rule}: a rule to determine direction of magnetic force on current-carrying conductor/moving charges%: let the forefinger of your left hand point in the same direction as the magnetic field, and your second finger in the direction of the electric current, then your thumb points in the same direction as the magnetic force acting on the current
			
			%\item \emph{magnetic forces between currents}: parallel/opposite currents attract/repel each other
			
			%\item \emph{magnetic force on moving charges}: a charged particle in a magnetic field experiences a magnetic force, which is normal to both the field and the direction of particle's motion
			
			\item \emph{motion of charged particles in uniform magnetic fields}: magnetic force acts at right angle to particle's velocity, so direction of motion keeps changing; uniform field so this force is constant; hence magnetic force provides centripetal force for circular/spiral motion
			
			\item \emph{velocity selector}: charged particles pass through a region where a uniform electric field and a uniform magnetic field are applied at right angles, if electric and magnetic forces on the particles are equal but opposite ($Eq=Bqv$), these particles moving at $v=\frac{E}{B}$ are not deflected
			
			\item \emph{cause of Hall effect}: charge carriers are deflected by magnetic force, causing an accumulation of electric charges on one side of the material, hence a voltage is established
			
			\item \emph{how a stable Hall voltage is established}: redistribution of charges causes an internal electric field, electric force on charge carriers opposes deflection of charge carriers due to magnetic force, the two forces tend to reach equilibrium, Hall voltage then stablises
			
			\item \emph{how to increase Hall voltage}: increase magnetic field/current (larger magnetic force, so greater p.d. needed to obtain equilibrium); use semi-conductor (smaller charge number density $n$); use a thin slice (smaller thickness $t$) ($V_H = \frac{BI}{ntq}$)
			
			\item \emph{specific charge}: ratio of electric charge of a particle to its mass
			
		\end{compactitem}
		
		\item {\large electromagnetic induction}
		
		\begin{compactitem}
			
			\item \emph{magnetic flux}: magnetic flux density $\times$  closed area at \underline{right angle} to the field ($\phi = BA$)
			
			\item \emph{magnetic flux linkage}: number of turns $\times$ magnetic flux density $\times$ closed area at \underline{right angles} to the field ($\Phi = NBA$)
			
			\item \emph{Faraday's law}: induced e.m.f. is proportional to the rate of change in magnetic flux (linkage) ($\mathcal{E}=\frac{\Delta \Phi}{\Delta t}$)
			
			\item \emph{Lenz's law}: induced e.m.f./current is always in a direction to produce \underline{opposing} effects to  the original change in magnetic flux
			
			\item \emph{eddy current}: when a metal disc/a bulk conductor experiences changes in magnetic flux/cutting of field lines, an e.m.f. is induced; if non-uniform cutting over whole disc/conductor,  different e.m.f. is induced in different parts, leading to eddy currents
			
			\item \emph{effects of induced currents}: magnetic force on induced currents opposes motion; induced currents dissipate energy as heat loss
			
			%\item \emph{eddy current braking}: mechanical energy of a system dissipates through heating caused by eddy currents, so the motion slows down and quickly comes to a stop 
			
		\end{compactitem}
		
		\item {\large alternating current}
		
		\begin{compactitem}
			
			\item \emph{alternating current}: a current whose direction changes with time
			
			\item \emph{r.m.s. voltage/current}: r.m.s. voltage/current of an alternating current equals the value of a steady voltage/current that produces the same \underline{mean} power
			
			\item \emph{principle of transformers}: alternating current in primary coils generates a changing magnetic flux, this flux is linked with secondary coils through transformer's iron core, according to Faraday's law, an output e.m.f. is induced in secondary coils
			
			%\item \emph{phase difference between input and output voltages}: input voltage in primary coils is proportional to magnetic flux in core, output voltage from secondary coils is proportional to rate of change in flux, so they are not in phase ($\Delta \phi = \frac{\pi}{2}$ if sinusoidal currents)
			%
			%\item \emph{energy loss in transformers}: heating in coils due to resistance; heating in iron core due to eddy currents; loss of magnetic flux
			%
			%\item \emph{usage of iron core in transformers}: strengthen magnetic flux (linkage) through the secondary coil/prevent loss of magnetic flux
			%
			%\item \emph{eddy currents in transformers}: the changing magnetic field in iron core can induce eddy currents, which cause heating
			%
			%\item \emph{how to reduce eddy currents}: laminate the iron core with insulate layers
			
			\item \emph{why transmitting electricity at high voltages}: high voltage means smaller current in cables (for same power output from power station), so less power loss during transmission
			
			\item \emph{advantages of alternating currents in electricity supply}: a.c. voltages can be easily changed by transformers; a.c. voltages can be easily created from generators %, we need high voltages to transmit electricity to reduce energy losses in transmission lines, but low voltages at receiving end for safety considerations
			
			\item \emph{rectification}: converting an alternating current into a direct current (using diodes)
			
			\item \emph{smoothing}: reduce ripples/fluctuations of output voltage (using capacitor)
			
			\item \emph{how to increase the amount of smoothing}: use larger capacitor for same load
			
		\end{compactitem}
		
		\item {\large quantum physics}
		
		\begin{compactitem}
			
			\item \emph{photon}: a packet/quanta of electromagnetic energy
			
			\item \emph{Einstein relation}: photon energy is proportional to frequency of radiation ($E=hf$)
			
			\item \emph{photoelectric effect}: when electromagnetic radiation shines upon a metal, electrons absorb photon energy and break free from metal surface
			
			\item \emph{threshold frequency}: minimum frequency of incoming radiation/photons to release electrons from metal surface
			
			\item \emph{work function (energy)}: minimum energy (of a photon) required for an electron to be emitted from surface of metal
			
			\item \emph{intensity of light}: power of radiation per unit cross sectional area, which is proportional to number of incoming photons per unit time and frequency of radiation ($I \propto N \times hf$)
			
			\item \emph{evidence provided by photoelectric effect for photon theory of light}: there exists a threshold frequency below which no electron is emitted; (maximum) kinetic energy of emitted electrons depends on frequency; electron emission rate depends on intensity of incident radiation; emission of electrons is immediate as soon as radiation is incident, etc.
			
			\item \emph{why emitted photoelectrons have a maximum kinetic energy}: electrons with maximum K.E. are released from the \underline{surface}, for electrons below the surface, additional energy is required to bring them to surface
			
			\item \emph{why emission of electrons is immediate}: interaction between photon and electron is one-to-one, so no time delay
			
			%\item \emph{electron-volt}: an unit of energy, one electron-volt equals the energy gained by an electron passing through a potential difference of one volt
			
			\item \emph{energy level (of electrons)}: electrons in an atom can only take certain values of energies, no intermediate state allowed
			
			\item \emph{relate line spectrum to discrete energy levels}: each line in the spectrum represents photon of a specific energy, photons are emitted/absorbed as a result of energy changes for electrons, specific energy changes so discrete/quantised energy levels for electrons
			
			\item \emph{emission spectrum}: electrons of an excited/hot gas transit/jump from high energy levels to low levels, photons with energies equal to change in electron energy levels are released, so only photons with specfic wavelengths/frequencies are emitted, giving a set of discrete bright lines
			
			\item \emph{absorption spectrum}: passing a white light through a cool gas, electrons absorb photons of correct energies to transit/jump from low energy levels to high levels,  only photons with specific wavelengths/frequencies are absorbed, these photons are re-emitted in all directions via de-excitation, giving a set of discrete dark lines in a continuous background
			
			%\item \emph{energy band}: interaction between neighbouring atoms in solid causes change of energy levels, each level splits into many sub-levels, huge number of atoms so many sub-levels spread into a band
			%
			%\item \emph{conductivity of metals}: conduction band of metal overlaps with valence band (no band gap), when temperature increases, number of charge carriers/free conducting electrons remains unchanged, but thermal lattice vibration increases, charge carriers/electrons are more likely to be scattered when they move around, so resistance of metal increases with temperature
			%
			%\item \emph{conductivity of NTC thermistors/LDR components}: at low temperature/in dark, conduction band is empty while valence band is fully filled, so no charge carrier to conduct electricity. as temperature/light intensity rises, valence electrons absorb energy to cross band gap and enter conduction band, there are free electrons in conduction band and holes are formed in valence band, number of charge carriers increases, so resistance of thermistor/LDR would decrease 
			
			% \item \emph{wave-particle duality}: all matter displays both wave-like and particle-like properties
			
			\item \emph{de Broglie wavelength}: wavelength of a particle moving with a certain momentum/speed ($\lambda = \frac{h}{p}$)
			
			\item \emph{electron diffraction}: passing an electron beam through a thin metal crystal, a set of concentric rings are observed on the fluorescent screen, such diffraction patterns show the wave-like behaviour of electrons
			
		\end{compactitem}
		
		\item {\large nuclear physics}
		
		\begin{compactitem}
			
			\item \emph{mass-energy equivalence}: mass is equivalent to energy ($E=mc^2$)
			
			\item \emph{binding energy}: minimum energy required to separate the nucleons (protons and neutrons) in a nucleus to \underline{infinity}
			
			\item \emph{mass defect}: difference between mass of a nucleus and total mass of its constituent nucleons at \underline{infinity}
			
			\item \emph{binging energy and stability of nucleus}: larger binding energy per nucleon means higher stability for nucleus (more difficult to break up this nucleus)
			
			\item \emph{binding energy and nuclear reaction}: a nuclear reaction is energetically possible if there is an increase in total binding energy after the reaction
			
			\item \emph{nuclear fission}: a large nucleus splits into two smaller nuclei of similar sizes and several neutrons
			
			\item \emph{nuclear fusion}: two light nuclei combine into a heavier/larger nucleus
			
			\item \emph{why fusion requires very high temperature}: for positively-charged nuclei to get close and fuse together, high kinetic energy is needed to overcome their electric repulsion, high K.E. so high temperature
			
			\item \emph{radioactive decay}: spontaneous and random emission of ionising radiation ($\alpha$-, $\beta$- or $\gamma$-radiation) from unstable nuclei
			
			\item \emph{spontaneity}: rate of decay is not affected by external conditions (such as temperature, pressure, etc.)
			
			\item \emph{randomness}: one cannot exactly predict when a particular nucleus will decay
			
			\item \emph{activity}: rate at which nuclei decay/number of decay events per unit time
			
			\item \emph{decay constant}: probability of a nucleus to decay per unit time
			
			\item \emph{half-life}: mean time taken for the number/activity of nuclei to decrease to half of the initial value
			
			\item \emph{count rate}: number of radioactive particles registered by a detector per unit time
			
			\item \emph{how count rate differs from activity of a sample}: count rate only measures a fraction of activity; there exists background radiation; radiation emitted may be absorbed by air/sample before reaching the detector; product nuclei may be also radioactive; there is dead time for the detector, etc.
			
		\end{compactitem}
		
		%\item {\large amplifier circuits}
		%
		%\begin{compactitem}
		%
		%\item \emph{properties of an ideal op-amp}: infinite open-loop gain; infinite input resistance/impedance; zero output resistance/impedance; no noise; infinite bandwidth; no time delay/infinite slew rate
		%
		%\item \emph{gain of op-amp}: ratio of output voltage to input voltage
		%
		%\item \emph{bandwidth of op-amp}: range of frequencies for input signals that have the same gain
		%
		%\item \emph{infinite slew rate}: changes in input voltage to op-amp cause immediate changes in output voltage, time delay is zero
		%
		%\item \emph{comparator}: a comparator compares the two voltages sent into the input terminals $V_+$ and $V_-$, and outputs a high or low voltage indicating which input is larger
		%
		%\item \emph{negative feedback}: a fraction of output voltage is sent back to the inverting input terminal
		%
		%\item \emph{effects of negative feedback}: reduce gain; increase gain stability; increase bandwidth of amplifier circuit; reduce distortion/noise, etc.
		%
		%\item \emph{virtual earth of inverting amplifier}: open-loop gain of op-amp is very high, if op-amp is not saturated, then $V_+ \approx V_-$. $V_+$ is earthed, i.e., $V_+=0$, so $V_-\approx0$ referred to as virtual earth
		%
		%\item \emph{relay}: an electromagnetic switch that can control a second circuit using small currents
		%
		% \end{compactitem}
	
	\item {\large medical imaging}
	
	\begin{compactitem}
		
		\item \emph{hardness of X-ray beam}: penetrating ability of beam, X-ray photons of higher energy/frequency have greater hardness
		
		\item \emph{production of X-ray}: electrons %emitted from a heated cathode and then
		accelerated through a high voltage strike into a target metal, when electrons decelerate, part of kinetic energy loss converts into X-ray photons
		
		\item \emph{how to increase the hardness of X-ray}: increasing the accelerating voltage: this leads to more energetic X-ray photons, so X-ray beam becomes more penetrating
		
		\item \emph{X-ray spectrum - braking radiation}: incident electrons bombarding target metal have a distribution of deceleration/K.E. loss, producing X-ray photons with a distribution of energies/frequencies, corresponding to a continuous background spectrum
		
		\item \emph{X-ray spectrum - characteristic radiation}: incident electrons excite electrons from inner shells of the target metal atoms, de-excitation of electrons from higher levels emits X-rays of particular frequencies, electron energy levels are discrete so this gives rise to a discrete characteristic spectrum
		
		\item \emph{why filtering out low-frequency X-rays}: low-frequency X-ray photons are not very penetrating and are mostly absorbed by patient's body, they do not contribute to imaging but might cause tissue damages, so should filter out using aluminium plates 
		
		\item \emph{sharpness}: ease with which edges of structures can be seen on an image
		
		\item \emph{contrast}: difference in blackening/colour/intensity for different structures on an image
		
		\item \emph{how to improve sharpness of X-ray images}: reduce the size of anode of the X-ray tube; collimate beam by a set of lead slits; use anti-scatter screen; let patient stay still, etc.
		
		\item \emph{how to improve contrast of X-ray images}: use contrast media (iodine-based media, barium meal, etc.) to turn the tissue into a better absorber of X-rays
		
		\item \emph{attenuation of X-ray/ultrasound}: decrease in intensity of X-ray/ultrasound as it passes through a substance
		
		\item \emph{absorption/attenuation coefficient}: as a \underline{parallel} beam of radiation/wave passes through matter, its intensity $I$ decreases as $I=I_0\mathrm{e}^{-\mu x}$, where $I_0$ is the initial incident intensity, $x$ is the distance the beam has travelled, $\mu$ represents the absorption/attenuation coefficient
		
		\item \emph{production of ultrasounds}: apply a high-frequency alternating voltage across a piezo-electric crystal of optimum thickness ($d=\frac{\lambda}{2}$), it vibrates and produces ultrasound waves with greatest intensity/amplitude
		
		\item \emph{detection of ultrasounds}: ultrasound exerts a varying pressure onto a piezo-electric crystal and generates a varying voltage across the crystal, this voltage can be measured
		
		\item \emph{(specific) acoustic impedance}: density of a medium $\times$ speed of sound wave travelling in that medium ($Z=\rho c$)
		
		\item \emph{impedance matching/why using gel in ultrasonic scans}: reflection coefficient for air-tissue boundary is nearly 100\%, waves are almost completely reflected and cannot enter patient's body, but at gel-tissue boundary, we have almost complete transmission, so transducer is coupled to skin using gel
		
		\item \emph{positron}: anti-particle of electron (same mass as electron, but carrying a charge of $+e$)
		
		\item \emph{annihilation}: when a particle meets its anti-particle, their mass-energy is converted into electromagnetic energy/photons
		
		\item \emph{why $\gamma$-photons produced in annihilation events have same energy but move in opposite directions}: conservation of total momentum requires sum of the momentum for the $\gamma$-photon pair must be zero, so two $\gamma$-photons carry equal but opposite momenta, equal momentum means equal energy, and opposite momentum means the two $\gamma$-photons travel at $180^\circ$ to each other
		
		\item \emph{tracer/radiotracer}: $\beta^+$-emitting radioactive nuclides attached/tagged to molecules that are taken up by human body
		
		\item \emph{line of response}: straight line drawn between the two detectors detecting the $\gamma$-photon pair produced by electron-positron annihilation
		
		%\item \emph{Larmor frequency}: frequency at which nuclei precess/rotate about axis of a magnetic field, this frequency is proportional to strength of the field applied $\omega_0 = \gamma B_0$
		
		\item \emph{principles of CT imaging, ultrasound imaging, PET imaging}: see appendix
		
	\end{compactitem}
	
	\item astrophysics
	
	\begin{compactitem}
		
		\item \emph{light-year}: distance travelled by light in vacuum in one year
		
		\item \emph{luminosity}: total radiation power output from an object/star
		
		\item \emph{radiant flux intensity}: power transmitted/received at right angles per unit area
		
		\item \emph{standard candle}: star with a known luminosity, can be used to determine the distance of galaxies
		
		\item \emph{examples of standard candles}: cepheid variable, type IA supernova
		
		\item \emph{black body}: an idealised object that absorbs all electromagnetic radiation
		
		\item \emph{black body radiation}: a black body emits electromagnetic radiation with a range of wavelengths (depending on its temperature)
		
		\item \emph{Wien's law}: wavelength at the peak intensity of a black body spectrum is inversely proportional to the surface temperature ($\lambda_\text{peak} T = \text{constant}$)
		
		\item \emph{redshift}: line spectra of stars/galaxies appear to have longer wavelengths due to their recession motion
		
		\item \emph{how redshift is determined}: spectrum of a star contains characteristic absorption lines that depend on its chemical composition, by comparing the spectral lines of the star with the absorption spectrum of the same element as measured in a lab, the amount of redshift can be determined
		
		\item \emph{Hubble's law}: recession speed of galaxies is proportional to the distance ($v=H_0 d$)
		
		\item \emph{Big Bang}: an event where the universe was created from an extremely hot and dense state
		
		\item \emph{evidence for the Big Bang Theory}: Hubble's law/expansion of the universe, cosmic microwave background radiation
		
		\item \emph{how Hubble's law leads to the Big Bang model}: Hubble's law suggests more distant galaxies are moving away at higher speeds, this implies an expanding universe, so in the past the galaxies used to be very close together, the universe should have started from a hot and dense state
		
		\item \emph{cosmic microwave background (CMB) radiation}: leftover/remnant radiation from the Big Bang that fills the entire universe: as universe expands, average temperature drops, and this should give a spectrum with peak intensity at the microwave band
		
	\end{compactitem}
	
	%\item {\large telecommunication}
	%
	%\begin{compactitem}
	%
	%\item \emph{amplitude/frequency modulation}: vary the amplitude/frequency of a high-frequency carrier wave in synchrony/phase with the displacement of an information signal
	%
	%\item \emph{sideband}: a carrier wave modulated by an information signal contains multiple frequencies other than the original carrier frequency
	%
	%\item \emph{bandwidth}: range of frequencies contained in a modulated carrier wave, for amplitude modulation, bandwidth equals twice the frequency of information signal ($B=2f_m$)
	%
	%\item \emph{advantages of FM transmission}: wider bandwidth, so better sound quality; more energy efficient; less affected by noise, etc.
	%
	%\item \emph{disadvantages of FM transmission}: wider bandwidth, so fewer radio channels available; more complex electronics, more complicated to design; shorter wavelength, so shorter transmission range, more transmitters needed so more expensive, etc.
	%
	%\item \emph{digital signal}: a signal containing only 0's and 1's with no intermediate values
	%
	%\item \emph{analogue-to-digital conversion (ADC)}: an analogue signal is sampled at regular time intervals and converted into binary numbers %for transmission
	%
	%\item \emph{parallel-to-serial conversion}: all the digits/bits of a number are produced from ADC at the same time, a parallel-to-serial converter then sends each digit/bit one after another
	%
	%\item \emph{regeneration}: original information signal is restored/reproduced, while noise/distortion is removed
	%
	%\item \emph{how to improve quality of reproduced signal}: use higher sampling frequency, so smaller step width/high-frequency components in original signal are detected; use greater number of bits for each sample, so smaller step height/smaller changes in voltages can be detected
	%
	%\item \emph{advantages of digital transmission}: can be easily regenerated, so less noise; can be encrypted, so better security; only take discrete values, so simpler and cheaper circuit; more information can be transmitted per unit time; extra data can be added to check for errors, etc.
	%
	%\item \emph{attenuation}: decrease in signal power/intensity/strength
	%
	%\item \emph{cause of attenuation}: power dissipation in heating due to resistance in electrical wires and cables; scattering and absorption by irregularities in optic fibres; energy absorption by medium for microwaves or radio waves, etc. 
	%
	%\item \emph{noise}: unwanted random interference that distorts the transmitted signal
	%
	%\item \emph{cause of noise}: induced voltages when nearby circuit is switched on or off; electromagnetic signals from mobile phones; lightning, or sparks from ignition; random thermal motion/vibration of atoms, etc.
	%
	%\item \emph{cross-linking}: a signal transmitted along a wire induces a copy in a nearby circuit
	%
	%\item \emph{co-axial cable}: a co-axial cable is made of an inner core wrapped by an outer conductor, the outer conductor is earthed so it provides an electromagnetic shielding for the inner core, noise/interference from outside is reduced
	%
	%\item \emph{optic fibre}: signals are transmitted in glass fibre through total internal reflection in form of infra-red pulses
	%
	%\item \emph{advantages of co-axial cables/optic fibres over wire-pairs}: less noise/distortion; less attenuation, fewer repeater amplifiers required; greater bandwidth, larger data capacity; higher security, difficult to be tapped, etc.
	%
	%\item \emph{uplink \& downlink for satellite communication}: uplink is greatly attenuated, downlink is amplified before being re-transmitted back to earth, so uplink and downlink use different carrier frequencies to avoid swamping/interference
	%
	%\item \emph{geostationary orbit}: an equatorial orbit where a satellite orbits from west to east with a period 24 hours (same period as rotation of earth) 
	%
	%\item \emph{advantage of geostationary satellites}: remain at relatively fixed position, so a satellite dish can always point towards a geostationary satellite; very high orbit, so large area of coverage, etc.
	%
	%\item \emph{disadvantages of geostationary satellites}: very high orbit so large time delay; can only cover equatorial regions and cannot cover areas at high latitudes, etc.
	%
	%\item \emph{advantages of polar satellites}: lower orbits so shorter time delay in communication; closer to earth, so suitable for surface observation; can cover whole globe, etc.
	%
	%\item \emph{disadvantages of polar satellites}: polar satellites are not stationary to ground observers, so they must be tracked; communication channels have to be swapped between several satellites for continuous communication, etc.
	%
	%\item \emph{measurement of attenuation in decibels}: attenuation calculated on logarithmic can turn very large ratios to smaller values, so easier to handle; total attenuation of several channels is additive if given in decibels
	
	%\item \emph{cellular network}: mobile phone sends signals to identify itself, base stations receive these signals and relay to the cellular exchange, computers at the cellular exchange select the base station with the strongest reception and assign a particular carrier frequency/time slot for the mobile phone
	%
	%\item \emph{why cellular network operates on ultra-high-frequency waves (UHF)}: transmission range of UHF is limited, then distant base stations can share the same carrier frequencies without interference; UHF has short wavelength, so convenient length of mobile handset aerial, etc.
	%
	%
	%\end{compactitem}
	
\end{enumerate}

\newpage 
\subsection*{Medical Physics Principles}

\subsubsection*{production of X-ray beams}

high-speed electrons accelerated in an electric field smash into an anode target metal

electrons decelerate and lose kinetic energy to give out X-ray photons

X-ray spectrum shows a few discrete spikes superimposed on a continuous background

there is a range of deceleration for electrons, so X-ray photons produced have a distribution of energies, so a continuous background spectrum (braking radiation)

orbiting electrons in metal atom can be knocked out, de-excitation of electrons from higher levels gives X-ray photons of specific energies, so a discrete line spectrum (characteristic radiation)


\subsubsection*{principle of CT (computerised tomography) scans}
patient’s body is considered to consist of many  2D slices, each slice contains many voxels

X-rays are incident from different angles for one slice, transmitted intensities are used to give information about each voxel with aid of a computer, 2D slice image is obtained

repeat the process, many 2D slice images obtained can be combined to form a 3D image

3D image can be rotated and viewed from different angles


\subsubsection*{production of ultrasound}

apply alternating voltages across piezo-electric crystal/quartz

crystal is made to vibrate and give out sound waves

if frequency of a.c. voltage is higher than 20,000 Hz, waves produced are ultrasounds

if crystal is cut into an optimum thickness (half wavelength of ultrasound), it can resonate to give ultrasound with maximum intensity

\subsubsection*{detection of ultrasound}

ultrasound exerts a varying pressure onto a piezo-electric crystal/quartz

a varying voltage is generated across the crystal, which is detected and measured

amplitude of this voltage gives information about intensity of the ultrasound

frequency of this voltage gives information about frequency of the ultrasound


\subsubsection*{principle of ultrasonic scans}
ultrasonic pulses produced from piezo-electric transducer are sent into patient’s body

reflected waves/echoes from boundaries of different media are received and detected by the same transducer, signals are processed and displayed

arrival times of reflected waves can give information about depth/thickness of structures

intensity of the reflected signals depend on acoustic impedance and attenuation coefficient of media, so can give information about nature of tissues

\subsubsection*{PET imaging}

tracer with $\beta^+$-emitter is injected into patient's body

tracer accumulates in certain organ/tissue, and emits positrons

positrons annihilate with nearby electrons to produces $\gamma$-photon pairs

$\gamma$-photons travel in opposite directions to each other

$\gamma$-photons are captured by a ring of detectors, line of response is established

difference in arrival time can be used to determine exact location of annihilation event

number of events per unit time gives information about concentration of the tracer


%\subsubsection*{MRI (magnetic resonance imaging)}
%
%apply a strong uniform magnetic field $B_0$, nuclei realign themselves in direction of the field
%
%nuclei precess/rotate about axis of field $B_0$ at Larmor frequency, which is in the region of radio frequency
%
%a non-uniform field is superimposed on $B_0$, Larmor frequencies at different parts of body will be different
%
%electromagnetic pulses $B_{RF}$ at Larmor frequency are sent towards patient, nuclei absorb energy, magnetic resonance is achieved
%
%nuclei then de-excite and emit radio frequency pulses, which are detected and processed
%
%relaxation time depends on environment of nuclei, so different tissues can be distinguished by different rates of relaxation
%
%frequency of pulse depends on the non-uniform field, so position of resonating nuclei can be located

}
