\section*{Appendix B. List of Formulae}
\addcontentsline{toc}{part}{Appendix B. List of Formula}
\rhead{ \fancyplain{}{LIST OF FORMULAE} }

\vspace*{-2em}
%{\setlength{\baselineskip}%
%	{4\baselineskip}
{\setstretch{2.4}
{
	\scriptsize

\begin{longtable}{p{.2\textwidth} p{.8\textwidth}}
	$\omega=2\pi f = \frac{2\pi}{T}$ & angular speed/frequency and period \\
	$v=\omega r$ & relating linear speed with angular speed  for circular motion\\ 
	$F_c=\frac{mv^2}{r} \text{ or } m\omega^2r$ & centripetal force required for an object undergoing circular motion\\
	$F=\frac{GMm}{r^2}$ & gravitational force between two masses at distance $r$, called Newton's law of gravitation \\
	$g=\frac{F}{m}$ & defining equation for gravitational field strength \\
	$g=\frac{GM}{r^2}$ & gravitational field strength due to point mass $M$ at distance $r$\\
	$E_p=-\frac{GMm}{r}$ & gravitational potential energy between two point masses separated by $r$ \\
	$\varphi=\frac{E_p}{m}$ & defining equation for gravitational potential \\
	$\varphi =-\frac{GM}{r}$ & gravitational potential at position $r$ due to mass $M$ \\
	$\frac{GMm}{r^2} = m\omega^2 r$ & gravity provides centripetal force for planets/satellites \\
	$\frac{1}{2}mv^2 - \frac{GMm}{r} = 0$ & to find minimum speed to escape from gravitational field to infinity \\
	$a=-\omega^2x$ & defining equation of simple harmonic motion (S.H.M.)\\
	$x=x_0\sin(\omega t+\phi)$ & variation of displacement of S.H.M. with time\\
	$v_\text{max}=\omega x_0$ & maximum speed of S.H.M. oscillator \\
	$v=\omega\sqrt{x_0^2-x^2}$ & speed-displacement relation for S.H.M. oscillator \\
	$\omega=\sqrt{\frac{k}{m}}$ & angular frequency of a mass-spring oscillator \\
	$E=\frac{1}{2}m\omega^2 x_0^2$ & total energy of S.H.M. oscillator \\
	$U=E_k+E_p$ & internal energy as sum of kinetic and potential energy \\
	$\Delta U = Q + W$ & first law of thermodynamics: change in internal energy equals heat and work done to system \\
	$Q=cm\Delta T$ & heat to raise temperature of substance without phase transition \\
	$Q=Lm$ & latent heat to change state of substance \\
	$W=p\Delta V$ & work done by/on gas at constant pressure \\
	$pV=nRT$ & ideal gas equation in terms of amount of substance $n$\\
	$pV=NkT$ & ideal gas equation in terms of total number of molecules $N$\\
	$N=nN_A=\frac{m}{M}N_A$ & number of molecules$N$, amount of substance $n$, mass of one molecule $m$ and molar mass $M$ \\
	$p=\frac{Nm\left<v^2\right>}{3V}$ & pressure of ideal gas determined by r.m.s. speed of particles \\
	$p=\frac{1}{3}\rho\left<v^2\right>$ & ideal gas pressure in terms of r.m.s. speed of particles and gas density \\
	$E_k=\frac{1}{2}m\left<v^2\right>=\frac{3}{2}kT$ & mean kinetic energy of particles at a particular temperature \\
	$F=\frac{Qq}{4\pi\epsilon_0r^2}$ & electric force between two point charges, called Coulomb's law \\
	$E=\frac{F}{q}$ & defining equation for electric field strength \\
	$ E = \frac{V}{d}$ & electric field strength between two oppositely-charge metal plates \\
	$E =\frac{Q}{4\pi\epsilon_0r^2}$ & electric field strength due to point charge $Q$ at distance $r$\\
	$E_p=\frac{Qq}{4\pi\epsilon_0r}$ & electric potential energy between two point charges separated by $r$ \\
	$V=\frac{E_p}{q}$ & defining equation for electric potential \\
	$V=\frac{Q}{4\pi\epsilon_0r}$ & electric potential at position $r$ due to point charge $Q$ \\
	$E=-\frac{\mathrm{d}V}{\mathrm{d}r}$, $g=-\frac{\mathrm{d}\varphi}{\mathrm{d}r}$ & electric/gravitational field strength equals negative potential gradient \\
	$C=\frac{Q}{V}$ & defining capacitance as ratio of charge to potential \\
	$C=C_1+C_2+\cdots$ & capacitance of a network of capacitors in parallel \\
	$\frac{1}{C}=\frac{1}{C_1}+\frac{1}{C_2}+\cdots$ & capacitance of a network of capacitors in series \\
	$C=4\pi\epsilon_0R$ & capacitance of a conducting sphere of radius $R$ \\
	$W=\frac{QV}{2} \text{ or } \frac{CV^2}{2}  \text{ or }  \frac{Q^2}{2C}$ & energy stored in a capacitor \\
	$I=I_0 \mathrm{e}^{-\frac{t}{RC}}$ & variation of current when charging/discharging a capacitor \\
	$V = V_0 \left( 1- \mathrm{e}^{-\frac{t}{RC}}\right)$ & increase in voltage when a capacitor is being charged \\
	$V = V_0 \mathrm{e}^{-\frac{t}{RC}}$ & drop in voltage when a capacitor is being discharged \\
	$F=BIl(\sin\theta)$ & magnetic force on current in straight wire (at angle $\theta$) to magnetic field \\
	$F=Bqv(\sin\theta)$ & magnetic force on moving charge (at angle $\theta$) in magnetic field \\
	$Bqv=\frac{mv^2}{r}$ & magnetic force provides centripetal force for particle moving in uniform magnetic field \\
	$qV=\frac{1}{2}mv^2$ & kinetic energy gain of charged particle through accelerating voltage \\
	$qE = Bqv$ & electric force on charged particles in equilibrium with magnetic force in velocity selector \\
	$\phi=BA(\cos\theta)$ & magnetic flux through cross-sectional area (at angle $\theta$ to normal of this area)\\
	$\Phi=NBA$ & magnetic flux linkage through coil with $N$ turns\\
	$\mathcal{E}=\frac{\mathrm{d}\Phi}{\mathrm{d}t}$ & Faraday's law, induced e.m.f. equals changing rate of flux (linkage) \\
	$\mathcal{E}=Blv$ & induced e.m.f. as wire cuts across magnetic field lines at certain speed \\
	$I=nAvq$ & current produced by charge carriers with number density $n$ at mean drift velocity $v$\\
	$V_H = Bvd$ & Hall voltage in a material with depth $d$ when charge carriers move with drift velocity $v$ \\
	$V_H = \frac{BI}{ntq}$ & Hall voltage across bulk conductor with thickness $t$ when a current flows through it\\
	$V=V_0\sin\omega t$ & an expression (not generic) for sinusoidal alternating voltage \\
	$V_\text{r.m.s}=\frac{V_0}{\sqrt{2}}$ & r.m.s. voltage of sinusoidal alternating current \\
	$P=\frac{V_\text{r.m.s}^2}{R} \text{ or } I_\text{r.m.s}^2R$ & mean power output of alternating current to load resistor\\
	$\frac{V_s}{V_p}=\frac{N_s}{N_p}$ & turns-ratio equation for transformers \\
	$I_pV_p=I_sV_s$ & current-voltage relation for ideal transformer (no power loss) \\
	$E=hf \text{ or } \frac{hc}{\lambda}$ & energy of single photon, called Einstein's relation \\
	$p = \frac{h}{\lambda}$ & momentum of a photon \\
	$hf=\Phi+E_{k,\text{max}}$ & photoelectric equation: photon energy = work function + K.E. of emitted electron\\
	$E_\text{high} - E_\text{low}=hf$ & energy level change of electrons and emission/absorption of photons \\
	$\lambda=\frac{h}{p}=\frac{h}{mv}$ & de Broglie wavelength of particle associated with momentum \\
	$E=mc^2$ & Einstein's mass-energy equivalence \\
	$E_B=\Delta mc^2$ & nuclear binding energy and mass defect \\
	$A=\frac{\Delta N}{\Delta t}$ & activity defined as rate of radioactive decay \\
	$A=\lambda N$ & activity related to decay constant \\
	$N=N_0\mathrm{e}^{-\lambda t}$ & number of undecayed nuclei decreases exponentially with time\\
	$A=A_0\mathrm{e}^{-\lambda t}$ & activity of radioactive sample decreases exponentially with time\\
	$t_{1/2}=\frac{\ln2}{\lambda}$ & half-life and decay constant \\
	%$V_\text{out}=G_0(V_+-V_-)$ & open-loop operational amplifiers \\
	%$G=\frac{V_\text{out}}{V_\text{in}}$ & voltage gain of amplifier circuits\\
	%$G=-\frac{R_\text{f}}{R_\text{in}}$ & gain of inverting amplifier \\
	%$G=1+\frac{R_\text{f}}{R_0}$ & gain of non-inverting amplifier \\
	$I=I_0\mathrm{e}^{-\mu x}$ & attenuation of X-rays/ultrasound through medium of distance $x$ \\
	$x_{1/2}=\frac{\ln2}{\mu}$ & half-value thickness and attenuation coefficient \\
	$Z=\rho c$ & acoustic impedance for a medium \\
	$\frac{I_r}{I_0} = \left(\frac{Z_1-Z_2}{Z_1+Z_2}\right)^2$ & ratio of reflected intensity to incident intensity at boundary of media \\
	$F = \frac{P}{A}$ & definition of radiant flux intensity \\
	$F = \frac{L}{4\pi d^2}$ & radiant flux intensity related to luminosity $L$ and distance $d$\\
	$\lambda_\text{peak} T = b$ & Wien's law, relating a star's spectrum to its surface temperature \\
	$L = \sigma T^4 \times 4\pi R^2 $ & Stefan-Boltzmann law, relating a star's luminosity to its surface temperature and its size\\
	$z = \frac{\Delta \lambda}{\lambda} = \frac{v}{c}$ & redshift of line spectra due to recession of stars/galaxies \\
	$v = H_0 d$ & Hubble's law, recession speed of galaxy is approximately proportional to the distance\\
	$t = \frac{1}{H_0}$ & estimation of age of universe
	%$\omega_0=\gamma B_0$ & Larmor frequency of nuclei procession in uniform magnetic field \\
	%$L_\text{dB}=10\log\left(\frac{P_1}{P_2}\right)$ & attenuation measured in logarithmic scale \\
	%$\text{SNR}_\text{dB}=10\log\left(\frac{P_\text{signal}}{P_\text{noise}}\right)$ & signal-to-noise ratio \\
	
\end{longtable} 

}\par}